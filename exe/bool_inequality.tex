\begin{displaymath}
    \textbf{pf} \in Id(Bool, true, false) \rightarrow N_0\ [\ ]\quad \equiv\quad \neg (true = false)
\end{displaymath}
Date le seguenti definizioni:
\begin{align*}
    f\quad \equiv&\quad El_{Bool}(z, (x).\hat{N_1}, (x).\hat{N_0}) \in U_0\ [z \in Bool]\\
    p_f\quad \equiv&\quad El_{Id}(w, (x).id(f(x))) \in Id(U_0, f(true), f(false))\ [w \in Id(Bool, true, false)]\\
    k\quad \equiv&\quad El_{Id}(w, (x).\langle \lambda y.y, \lambda y.y \rangle) \in T(x) \leftrightarrow T(y)\ [x \in U_0, y \in U_0, w \in Id(U_0, x, y)]
\end{align*}
La dimostrazione è basata sull'utilizzo degli universi, altrimenti non potremmo mai distinguere termini diversi all'interno della nostra teoria dei tipi.
La funzione $f$ propaga elementi diversi all'interno di universi distinti per poterne poi effettuare l'uguaglianza.
La funzione $k$ invece crea l'\textit{isomorfismo} necessario per distinguere $\hat{N_1}$ da $\hat{N_0}$.\\
\texttt{dovrei chiarire quando le procedure sono astrazioni oppure no per essere formale}
